\documentclass[10pt, onecolumn]{article}
\usepackage{fullpage}
\usepackage{amsmath}
\usepackage{amssymb}
\usepackage[dvips]{graphicx}
\usepackage{url}
\usepackage{hyperref}

%opening
\title{Kismet Decision Engine}
\author{Alan Meert}
\date{7 Feb 2014}


\begin{document}
\maketitle

The goal of the kismet decision engine is to come up with a decision process that is optimal for playing kismet. 
In any roll, there are 252 unique dice combinations that can appear (each with different probabilities of appearing).
The possible combinations are:
\begin{itemize} 
\item Group 5: 6 
\item Group 4-1: 30 
\item Group 3-2: 30 
\item Group 3-1-1: 60 
\item Group 2-2-1: 60 
\item Group 2-1-1-1: 60 
\item Group 1-1-1-1-1: 6 
\end{itemize}
where the numbers reffer to the number in each group of non-unique die. For instance, group 4-1 would be 4 of a kind in kismet. 
One can deduce the probability of a specific die combination by considering the number of permutations of that combination available and dividing
that by the total number of dice combinations ($6^5=7776$). So the probabilities are:
\begin{itemize} 
\item Group 5: P$=6*(5!/5!)/7776 \approx  0.00077$
\item Group 4-1: P$=30*(5!/4!)/7776 \approx  0.01929$ 
\item Group 3-2: P$=30*(!5/(2!3!))/7776 \approx  0.03858$ 
\item Group 3-1-1: P$=60*(5!/3!)/7776 \approx 0.07716$ 
\item Group 2-2-1: P$=60*(5!/(2!2!))/7776 \approx  0.11574$ 
\item Group 2-1-1-1: P$=60*(5!/2!)/7776 \approx  0.23148$ 
\item Group 1-1-1-1-1: P$=6*5!/7776 \approx  0.46296$ 
\end{itemize}
 
\end{document}
