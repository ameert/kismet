\documentclass[10pt, onecolumn]{article}
\usepackage{fullpage}
\usepackage{amsmath}
\usepackage{amssymb}
\usepackage[dvips]{graphicx}
\usepackage{url}
\usepackage{hyperref}

%opening
\title{Kismet Decision Engine}
\author{Alan Meert}
\date{7 Feb 2014}


\begin{document}
\maketitle

The goal of the kismet decision engine is to come up with a decision process that is optimal for playing kismet. 
In any roll, there are 252 unique dice combinations that can appear (each with different probabilities of appearing).
The possible combinations are:
\begin{itemize} 
\item Group 5: 6 
\item Group 4-1: 30 
\item Group 3-2: 30 
\item Group 3-1-1: 60 
\item Group 2-2-1: 60 
\item Group 2-1-1-1: 60 
\item Group 1-1-1-1-1: 6 
\end{itemize}
where the numbers refer to the number in each group of non-unique die. For instance, group 4-1 would be 4 of a kind in Kismet. 
One can deduce the probability of a specific die combination by considering the number of unique permutations of that 
combination available and dividing that by the total number of dice combinations ($6^5=7776$). So the probabilities are:
\begin{itemize} 
\item Group 5: P$=6*(5!/5!)/7776 \approx  0.00077$
\item Group 4-1: P$=30*(5!/4!)/7776 \approx  0.01929$ 
\item Group 3-2: P$=30*(!5/(2!3!))/7776 \approx  0.03858$ 
\item Group 3-1-1: P$=60*(5!/3!)/7776 \approx 0.07716$ 
\item Group 2-2-1: P$=60*(5!/(2!2!))/7776 \approx  0.11574$ 
\item Group 2-1-1-1: P$=60*(5!/2!)/7776 \approx  0.23148$ 
\item Group 1-1-1-1-1: P$=6*5!/7776 \approx  0.46296$ 
\end{itemize}
 
In addition to the 252 unique dice combinations, after each roll, the player can choose to keep between 0 and 5 dice and reroll. 
For each dice combination, this gives at most $1+5+10+10+5+1=32$ options at each turn, one of which is to score the dice on the card.
However, many of these choices are not unique. Choices to keep or discard dice also cause different hands to ``converge'' to the same
hand going forward. After accounting for the non-unique combinations, there are 210 combinations that continue the hand 
(i.e. at least on die is dicarded and rerolled). There are then an additional 252 combinations that stop the hand and score. 

After each roll, the task for the computer is then to decide among several options:
\begin{enumerate}
 \item keep between 0 and 4 dice
 \item stop the hand and score
\end{enumerate}

If the algorithm decides to stop, then it must also decide what to score the hand as. This should also be a consideration of the
algorithm when deciding how to move forward in the hand.

\end{document}
